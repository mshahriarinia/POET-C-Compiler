\documentclass[11pt]{article}
\usepackage{makeidx}
\usepackage{appendix}
\usepackage{ifthen}
\usepackage{listings}
\usepackage{verbatim}
\usepackage[dvips,pdfpagemode=UseOutlines,bookmarks,bookmarksopen,
pdfstartview=FitH,colorlinks,linkcolor=blue,
pdftitle={POET Language Manual},
pdfauthor={Qing Yi},
citecolor=blue,urlcolor=red]{hyperref}
\textwidth 6.5in
\textheight  8.8in
\oddsidemargin  0in
\evensidemargin 0in
\topmargin      -0.3in
\parindent      0.25in

%\def\bs{\char'134 } % backslash in \tt font.
%\newcommand{\ie}{i.\,e.,}
%\newcommand{\eg}{e.\,g..}
%\DeclareRobustCommand\cs[1]{\texttt{\char`\\#1}}
\makeindex 

\title{POET Tutorial: Building A Small Compiler Project}
\author{ Qing Yi \\
  University of Texas At San Antonio \\(qingyi@cs.utsa.edu) }

\begin{document}

\maketitle
%%%%%%%%%%
\section{Setting Up The Translation}
\label {sec-driver}

A compiler is essentially a translator that reads in some some code in an input language and output 
the result in another language.
The following illustrates the structure of a compiler in POET that translates between two 
 languages.
\begin {verbatim}
include utils.incl
<********************* A compiler/translator project *******************>
< parameter inputFile type=STRING default="" message="input file name" />
< parameter outputFile type=STRING default="" message="output file name" />

<* declares the start non-terminal of your grammar *>
<code Goal/> 

<* read the inputFile, parse it using grammar syntax defined in file "InputLang.code",
    and save the parsing result (the abstract syntax tree of the input code)
    into variable MyAST *>
<input syntax="InputLang.code" parse=Goal from=inputFile to=MyAST />

<*  define functions/routines to accomplish your task *>
<xform TranslateCode pars=(input)>
......
</xform>

<* perform whatever additional operations here *>
<eval  result=TranslateCode(MyAST) />   

<* output your result to outputFile *>
<output syntax="OutputLang.code" to=outputFile from=result/>
<********************** End of my Code ******************************>
\end {verbatim}

The first line is an include command, which reads the $utils.incl$ file (located in POET/lib directory)
before reading any instructions in the current file.
{\bf \noindent NOTE: The include commands must be at the start of a POET program to be recognized.
If appearing in mid of a program, they will be ignored (treated as source code of an input language).}


The following two lines are $parameter$ declarations that define the values of $inputFile$ and $outputFile$.
Specifically, each $parameter$ keyword declares a global variable whose value can be modified through command-line options. For example, 
\begin {verbatim}
<parameter inputFile type=STRING default="" message="input file name" />
\end {verbatim}
declares a global variable named $inputFile$. 
In particular, the variable should contain values of type $STRING$.
The default value of the variable is empty string (``"). 
And the meaning of the variable is to store ``input file name". Similarly, the $outputFile$ is a 
$STRING$ variable that contains the output file name.
The command-line to invoke the Identity translator therefore is the following (assuming
the current directory is $POET/examples$.
\begin {verbatim}
> ../src/pcg [-pinputFile=<myInputFile>] [-poutputFile=<myOutputFile>]  MyTranslator.pt
\end {verbatim}
Here each command-line option $-p<var>=<value>$  specifies a new value for a global variable
used in the $MyTranslator.pt$ File. The capability allows the translator to
read input from an arbitrary file and write output to another arbitrary file. All the command-line options
are optional --- when no value is given for a global parameter, the parameter contains its default value (declared using the $default$ keyword within the $parameter$ declaration). 

Following the $parameter$ declarations is a forward declaration that declares
$Goal$ as the name of a code template. This code template name should be defined
in the file that contains grammar syntax definitions for the input file, assumed to be 
the $inputLang.code$ file in this case. 

\vspace{5mm}
{\bf \noindent NOTE: 
Except for the $include$ commands at the start of the program, all the commands in a POET program are first read in and saved in an internal
representation before being evaluated. 
In particular, the $input$ commands in the mid of a program are not evaluated until all the other commands and 
declarations have been read. Because of this, although the name $Goal$ is defined in
$InputLang.code$, it needs to be declared in $MyTranslator.pt$ as a code template name to avoid
being treated as a regular global variable.}
\vspace{5mm}

Following the forward $code$ declaration for $Goal$ is an input command.
Specifically, the $input$ command opens the input file (name specified by the $inputFile$ variable),
reads in the input code (the content of the input file), and saves the input code as value of the global
variable $MyAST$. The parsing of input code is based on the grammar definition contained in file $inputLang.code$ (enforced by the $syntax="inputLang.code"$ specification). Additionally, the
code template $Goal$ will be used as 
the start non-terminal (the top-level variable) of the grammar to parse the input code
 (enforced by the $parse=Goal$ specification). 

After the input command,  is a $xform$ declaration that defines a function/routine named
$TranslateCode$, which accomplishes whatever task your compiler may need to accomplish.
Specifcally, it uses all the capabilities that POET provides (e.g., pattern matching, AST traversal,
and program transformations) to analyze or transform the input code. 
{\bf \noindent NOTE: By default, all names in POET are assumed to be regular variable names unless they
have already appeared as the name of a code template or xform routine declaration. For more
details, see the language manual. }


Following the $xform$ routine definitions, an $eval$ command invokes the $TranslateCode$ routine 
with $MyAST$ as actual parameter. 
The result of language translation is then saved in a global variable $result$. 
The $eval$ command is used to evaluate expressions and statements at the global scope
in order to compute the final output. 
{\bf NOTE: All POET expressions must be embedded within an eval command to be evaluated at the
global scope.}

After reading the input code and applying transformations, the last $output$ command writes the 
compilation result (contained in global variable $result$) to an external file whose name specified by the $outputFile$ variable. In particular, 
the result is unparsed based on the grammar definition contained in file $OutputLang.code$ (enforced by the $syntax=inputLang$ specification).
If there is no syntax specification,
the result will be output as a sequence of integer/string
tokens.

\vspace{5mm}
{\bf \noindent NOTE:} when the $inputFile$ variable contains an empty string, the input code will be read from
the standard input (the user will be prompted to type in the input code); when the $outputFile$ variable
contains an empty string, the input code will be written to the standard output (the screen).

\vspace{5mm}
{\bf \noindent NOTE:} as illustrated by the $MyAST$ variable, global variables in POET can be used without being declared.  Command-line parameters are also considered as global variables.
Although the global $parameter$ variables can have their types specified (declared),
the specification is optional (not required).
\vspace{5mm}

\section {Writing Grammar Specifications For Parsing}
\label {sec-grammar}

POET supports automatic parsing of arbitrary languages as long as the grammar syntax of
the language is defined in a separate syntax file. The specification of grammar syntax is
through code templates in POET. There is a direct correlation between specifying the
grammar syntax using BNF and using POET. In particular, suppose we have the following
specification of grammar syntax via BNF.

\begin {verbatim}
Goal ::= Exp
Exp ::= Exp + Factor | Factor
Factor ::= id | intval
\end {verbatim} 

After eliminating left-recursion (POET parses the input code in a top-down recursive descent fashion
and therefore left-recursion in the grammar needs to be eliminated), we obtain the following grammar.

\begin {verbatim}
Goal ::= Exp
Exp ::=  Factor Exp'
Exp' ::= + Factor Exp' | epsilon
Factor ::= id | intval
\end {verbatim} 

The above grammar can be expressed using the following POET code template definitions.
\begin {verbatim}

<code Exp/>
<code Exp1/>
<code Exp2/>
<code Factor />

<code Goal pars=(content : Exp) >
@content@
</code>

<code Exp pars=(opd1 : Factor, rest : Exp1 )>
@opd1@ @rest@
</code>

<code Exp1 pars=(content : Exp2 | "") >
@content@
</code>

<code Exp2 pars=(opd2: Factor, rest : Exp1)>
+ @opd2@ @rest@
</code>

<code Factor pars=(content : Name | INT)>
@content@
</code>
\end {verbatim}
Here an extra code template $Exp2$ is used to specify the first BNF production 
for Exp1 (i.e., $Exp' ::= + Factor Exp'$), where
Exp1 corresponds to non-terminal $Exp'$ in the BNF specification.
Specifically, the {\bf name} of each code template corresponds to the left-hand non-terminal of each 
BNF production, and the {\bf body} of each code template corresponds to the right-hand side of each
BNF production. If any symbol in the right-hand side of a BNF production is itself a non-terminal
or has a value that needs to be saved, a variable is created as parameter of the code template, 
and the variable (surrounded by a pair of @'s) is used inside the code template body instead.
For example, the following code template definition corresponds to the BNF production
$Exp ::= Factor Exp'$.
\begin {verbatim}
<code Exp pars=(opd1 : Factor, rest : Exp1 )>
@opd1@ @rest@
</code>
\end {verbatim}
Here two variables, $opd1$ and $rest$ are created for the non-terminals $Factor$ and $Exp'$ 
in the original production respectively. 

POET code templates are not only used to define grammar syntax of languages, they are also
used to define the internal representation of the input code using an abstract syntax tree. 
Suppose the following POET program is used to parse an arbitrary input code (see Section~\ref {sec-driver}).
\begin {verbatim}
include utils.incl

<parameter inputFile message="input file name"/>
<parameter outputFile  message="output file name"/>
<parameter inputlang   message="input syntax"/>

<code Goal/>

<input from=(inputFile) syntax=(inputlang) parse=Goal to=inputCode/>

<eval PRINT(inputCode)/>

\end {verbatim}
Here $PRINT(inputCode)$ prints out the AST constructed from parsing the input code. 
If the POET code templates above is used to define grammars in the  $inputlang$ file, 
and if an expression $a+b+5$ is used as the input code, the following AST will be printed out.

\begin {verbatim}
Goal#Exp#(
    Factor#"a",
    Exp1#Exp2#(
        Factor#"b",
        Exp1#Exp2#(
            Factor#3,
            Exp1#"")))
\end {verbatim}

%%%%%%%%
\section {Rebuilding AST}
\label {sec-AST}
%%%%%%%%%
Unless specified otherwise, POET automatically builds an AST node (code template object) for
each code template definition used in the parsing process. To build a more elegant AST, 
you can reconfigure the parsing process by defining a $rebuild$ attribute for some code templates.
For example, the context-free grammar definitions in section~\ref {sec-grammar} need to be 
extended with the following attribute grammar to build an AST.
\begin {verbatim}
Goal ::= Exp       { Goal.ast = Exp.ast; }
Exp ::=  Factor  {Exp'.inh = Factor.ast; } Exp'  { Exp.ast = Exp'.ast; }
Exp' ::= + Factor { Exp'1.inh = BinaryOperator("+", Exp'.inh, Factor.ast); } Exp'1 
        {  Exp'.ast=Exp'1.ast;}
        | epsilon { Exp'.ast = Exp'.inh; }
Factor ::= id {Factor.ast = Variable(id.name); } 
        | intval   { Factor.ast = ICONST(intval.val); }
\end {verbatim}
The above attribute grammar can be translated to the following POET implementations.
\begin {verbatim}
include utils.incl

<code Exp/>
<code Exp1/>
<code Exp2/>
<code Factor />
<xform BuildExp pars=(inh, content)   />

<code Goal pars=(content : Exp) rebuild=content >
@content@
</code>

<code Exp pars=(opd1 : Factor, rest : Exp1 ) rebuild=BuildExp(opd1,rest)>
@opd1@ @rest@
</code>

<code Exp1 pars=(content : Exp2 | "") rebuild=content >
@content@
</code>

<code Exp2 pars=(opd2: Factor, rest : Exp1) >
+ @opd2@ @rest@
</code>

<code Factor pars=(content : Name | INT) rebuild=content>
@content@
</code>

<code Bop pars=(op : STRING, opd1 : Exp, opd2 : Exp)>
(opd1 op opd2)
</code>

<xform BuildExp pars=(inh, content)>
switch (content) {
  case Exp2#(opd2, rest) : BuildExp(Bop#("+", inh, opd2), rest)
  case "" : inh
 }
</xform>
\end {verbatim}
{\bf Note that when using code templates to automatically parse input programs, POET
supports only  synthesized attributes during parsing.} In particular,
each $rebuild$ attribute of a code template holds an expression 
that defines what AST node to return as the result  of parsing the given code template.
After POET successfully parses the input program 
using each code template, it evaluates the corresponding $rebuild$ attribute
using all the known parameters and attributes of the code template. 
For example, in the above POET code, the $BuildExp$ routine is invoked in the $rebuild$
attribute of code template $Exp$ whenever the $Exp$ template has been
used to successfully parse a fragment of the input code. 
 
 Because the POET built-in code template parser provides limited support for inherited attributes
 (you can only define a single INHERIT attribute, which holds the previous code template object built
 from input,  for each code template),  it is more convenient to
 convert inherited attributes into synthesized attribute evaluation. In particular, you can
 save all the synthesized attributes relevant for building an AST until both the synthesized
 and inherited attributes can be collectively used.
For example, the $BuildExp$ routine is invoked to  the rebuild function for $Exp$
instead of $Exp1$ because the inherited attribute for $Exp1$ is not available until the parsing
of $Exp$ has succeeded. 

%%%%%%
\section {Building A Symbol Table}
\label {sec-table}
%%%

POET provides a compound data type, MAP, to support the need of associative tables that map
a collection of values to their attributes. Each symbol table is an associative map that associate
type information with variable names. Suppose the following code templates define the data 
structure for storing type information.
\begin {verbatim}
<code IntType/>
<code FloatType />
<code ArrayType pars=(elemType, arraySize) />
\end {verbatim}
The following POET statement constructs an empty symbol
table and stores it into a local variable $table$.
\begin {verbatim}
table = MAP(Name,IntType|FloatType|ArrayType);
\end {verbatim}
The following statements check whether a variable named $x$ is already in in the symbol
table; if not, it inserts $x$ into the symbol table as a $IntType$ variable. This can be used
as an action for type checking when seeing type declarations such as ``int x,y,z;". 
\begin {verbatim}
if (table[x] != "")
     ERROR( "variable multiply declared: " x);
else table[x] = IntType;
\end {verbatim}

The following statements insert three integer variables, ``a", ``b", and ``c", into the symbol table.
\begin {verbatim}
   foreach ( ("a","b","c") : (x=STRING) : TRUE) {
     if (table[x] != "")
        ERROR( "variable multiply declared: " x);
     else table[x] = IntType;
   }
\end {verbatim}
To find out more about the use of the $foreach$ loop and associative tables, see POET manual.

If nested scopes are allowed, a list of symbol tables can be used to represent the context information
of each block. The following POET routine can be invoked to determine the type information of 
a variable in a nested block.
\begin {verbatim}
<xform LookupVariable pars=(symTableList, varName)>
if (symTableList == "") 
    ERROR("Variable not found: " varName);
curTable = HEAD(symTableList);
res = curTable[varName];
if (res != "") RETURN res;
LookupVariable(TAIL(symTableList), varName)
</xform>
\end {verbatim}

%%%%%
\section {Type Checking}
\label {sec-type}
%%%
To apply type checking to an input program, the AST representation of the program
must be traversed, and a type must be determined for each expression within the program. 
For example, suppose we have the following translation scheme for type checking.
\begin {verbatim}
Exp ::= {Exp1.symTable=Exp.symTable;} Exp1 + {Exp2.symTable=Exp.symTable; } Exp2 
    { if (Exp1.type == INT && Exp2.type == INT) Exp.type = INT;
       else if (Exp1.type == FLOAT && Exp2.type==FLOAT) Exp.type=FLOAT;
       else ERROR; }    
  | id { Exp.type = lookup_type(Exp.symTable,id); }  
  | intval { Exp.type = INT; }
\end {verbatim}
It can be implemented using the following POET function.
\begin {verbatim}
<xform TypeCheckExp pars=(symTable, exp)>
  switch(exp)
  {
    case Bop#("+", exp1, exp2):
          type1 = TypeCheckExp(symTable, exp1);
          type2 = TypeCheckExp(symTable, exp2);
          if  (type1 : IntType && type2 : IntType)  returnType=IntType;
          else if (type1 : FloatType && type2 : FloatType) returnType=FloatType;
          else ERROR("Type checking error: " exp);
          returnType
     case STRING:  symTable[exp]
     case INT : IntType
  }
</xform>
\end {verbatim}

The type information of each expression may be additionally saved in the symbol table of each scope, so that the information can be used for later code generation and optimizations. This is implemented
by the following POET routine.
\begin {verbatim}
<xform TypeCheckExp pars=(symTable, exp)>
  switch(exp)
  {
    case Bop#("+", exp1, exp2):
          type1 = TypeCheckExp(symTable, exp1);
          type2 = TypeCheckExp(symTable, exp2);
          if  (type1 : IntType && type2 : IntType)  returnType=IntType;
          else if (type1 : FloatType && type2 : FloatType) returnType=FloatType;
          else ERROR("Type checking error: " exp);
          symTable[exp] = returnType;   <<* saving the type of exp in symbol table
          returnType
     case STRING:  (symTable[exp])
     case INT : IntType
  }
</xform>
\end {verbatim}
Note that in POET, the last expression evaluated is the value returned by a xform routine. 
For details on how to define {\it xform} routines and how to use the pattern pattern capabilities
in POET, see the language manual.

\section {Building A Control Flow Graph }

There are two steps involved in building a control flow graph: identifying basic blocks, and 
using the basic blocks as nodes to build an arbitrary graph. Because each basic block is
a sequence of statements or expressions, you can implement each basic block using a
list of AST nodes. Suppose you have already identified a set of statements/expressions that
should be included as components of basic blocks and  a subset of these ASTs that
serve as the starting points of different basic blocks.
The following POET function can be invoked to build a list of all the basic blocks 
(Need poet.1.02.04. see POET/examples/Anal.pt).

\begin {verbatim}
<code BasicBlock pars=(label, stmts)>
@label@ : @stmts@
</code>

<xform BuildBasicBlocks pars=(inclASTs : MAP(_,INT), input)>
basicBlocks = "";
curBlock = "";
count=0;
foreach_r (input :  (cur=_) : FALSE) {
  switch (inclASTs[cur]) {
  case "": CONTINUE;
  case 1: curBlock = cur :: curBlock;
  case 2: curBlock = cur :: curBlock;
            count = count + 1;
            basicBlocks = BuildList(BasicBlock#(count, curBlock), basicBlocks);
            curBlock = "";
  default: ERROR("Unrecognized map value : " inclASTs[cur] " : for AST : " cur);
  }
}
basicBlocks
</xform>
\end {verbatim}
In the above code, we define a code template $BasicBlock$ to be composed of a label together
with a sequence of statements.  A function $BuildBasicBlocks$ is then defined to build all the
basic blocks.
In particular, we use an associative map (the $inclASTs$ parameter of the $BuildBasicBlocks$ function)
to remember which AST nodes should be included as part of a control flow graph.  
For each AST node $cur$ within the given code $input$, if $cur$ is included in $inclASTs$
(that is, $inclASTs[cur] != ""$), then $cur$ should be included in the current basic block being built;
otherwise, it is simply skipped.
If $inclASTs[cur] == 1$, then the current AST node $cur$ does not start a new basic block,
and we continue to add more statements into the current basic block. However, if
$inclASTs[cur] == 2$, then the current AST node $cur$ starts a different basic block, so we include
$cur$ as the first statement of the current basic block and then starts a new one.
Note that the $input$ code is traversed in the reverse order, so the first statement of each basic
block is the last one added to the basic block. After the $foreach\_r$ loop, the resulting
$basicBlocks$ is returned as result of the $BuildBasicBlocks$ function.
For details on the POET statements (e.g., $foreach\_$,
$switch$), lists (e.g., the $::$ operator), and  associative map (e.g., the $[]$ operator), see POET
language manual.

The second step of control-flow graph construction is to connect the basic blocks with edges.
The following defines code templates that support the construction and output of a control-flow
graph.
\begin {verbatim}
<code GraphEdge pars=(from,to) >
"@from@"->"@to@"
</code>
<code GraphEdges pars=(content) parse=LIST(GraphEdge,"\n")/>
<code CFG pars=(label, blocks, flow : GraphEdges)>
digraph @label@
{
  @flow@
}

</code>
\end {verbatim}
The following function adds a control-flow edge between each pair of adjacent basic blocks in a list
and then output the generated control flow graph.
\begin {verbatim}
<xform BuildCFG pars=(blocks)>
   edges = "";
   from = HEAD(blocks);
   for ( p_blocks = TAIL(blocks); p_blocks != ""; p_blocks = TAIL(p_blocks))
   {  
      edges = BuildList( GraphEdge#(from, HEAD(p_blocks)), edges);
      from = HEAD(p_blocks);
   }   
   CFG#("CFG",blocks,edges)
</xform>
\end {verbatim}
Here the $from$ variable is used to keep track of the basic block to flow from, and the loop uses
a variable $p\_blocks$ to traverse through a list of blocks (the $blocks$ parameter) while adding 
a $GraphEdge$ between each pair of adjacent basic blocks. The following is an example output
of the constructed control-flow graph.
\begin {verbatim}
digraph CFG
{
  "2[int i,j,l;C[j*ldc+i] = beta*C[j*ldc+i];]"
   ->
   "1[C[j*ldc+i] = C[j*ldc+i]+alpha*A[i*lda+l]*B[j*ldb+l];]"
}
\end {verbatim}
The graph can be output into a file named, say $output.dot$, and then viewed using $graphviz$,
the open-source graph visualization project from AT\&T research (you can download the software
for free by googling $graphviz$ or $dot$).

\section {Traversing The Control-flow Graph}
Building a control-flow graph simply means identifying all the basic blocks and then
adding a sequence of graph edges to represent the flow of control among the basic blocks.
However, control-flow graph often needs to be traversed in efficient ways. 
For example, to support data flow analysis,  we need to quickly find all the successors 
(or predecessors) of an arbitrary basic block. 

Because POET was designed to target code transformation instead of program analysis,
the code templates in POET can be used to easily build flexible tree data structures (e.g., the AST)
but currently does not support cyclic data structures. 
Therefore, you need to use additional data structures, e.g., associative maps, to compensate
such limitation. For example,  to quickly find the successors of an arbitrary basic block,
the following code builds an associative table that maps each basic block to a list of its successors.
\begin {verbatim}
<xform MapSuccessors pars=(cfgEdges)>
  res = MAP(_,_);
  foreach (cfgEdges : GraphEdge#(CLEAR from, CLEAR to) : TRUE)
  {
     res[from] = BuildList(to, res[from]);
  }
  res
</xform>

\end {verbatim}

\section {Data-flow Analysis}
When performing data-flow analysis, each basic block needs to be associated with a set
of information (e.g., a set of expressions or variables). These sets of information can be implemented
using either the built-in compound type $lists$ in POET or using associative maps in POET. 
In general, you need to first traverse the entire input, find the set of all the entities in your analysis
domain. The set of information associated with each basic block is then a subset of this domain.

Since set intersection, union, and subtraction are often applied frequently to subsets of the analysis
domain, it is often convenient to implement each subset of the analysis domain using the POET associative map. In particular,
if an item is mapped to the empty string (i.e., ``") in the associaitive map, then the item is not in the subset; otherwise, it is in the subset.

\section {Modifying the AST}
\label {sec-mod}
The $REPLACE$ operator is the most useful operation in POET to support code transformation.
In particular, the $REPLACE$ operator can be invoked using two different syntax, as illustrated
int the following.
\begin {verbatim}
    REPLACE("x","y", SPLIT("","x*x-2"));
    REPLACE( (("a",1) ("b",2) ("c",3)), SPLIT("","a+b-c"));
\end {verbatim}
The first REPLACE invocation replaces all occurrences of string ``x" with string ``y" in the third argument
of $REPLACE$ (the result of $SPLIT(``",``x*x-2")$); the second REPLACE invocation makes a sequence
of replacement. In particular, it first looks for the occurrence of string ``a". When found, it replaces ``a"
with 1, and then moves on and tries locate string ``b", the origin of the second replacement.
Once the string ``b" is found, it is replaced with 2 and then we proceeds by trying to look for ``c".
Each replacement is applied only once, and a warning message is issued if the entire list of
replacement specifications are not exhausted when reaching the end of the input argument (the last argument of REPLACE).

Note that each REPLACE operation performs the replacements by returning a new data structure.
The original input (i.e., the last arguments of the operation) is never modified (unless a trace handle
is used, which is out of the scope of this discussion).
More generally, POET is a functional programming language.  
Except for tracing handles (which are more advanced topics and are not discussed
here), none of the compound data structures in POET can be modified. 
Therefore, during the process of redundancy elimination, you should try to build a list of all the 
replacement  and then apply all the replacements in the end. The following is an example
illustrating this process (see POET/examples/C2F.pt)
\begin {verbatim}
  repl="";
  foreach_r (input : (VarTypeDecl#(CLEAR type,_)) :FALSE)
     { repl=BuildList( (type, TranslateCtype2Ftype(type)), repl); }
  input = REPLACE(repl, input);
\end {verbatim}
Here to translate variable type declarations in C syntax to those in Fortran, the above code
first build an empty replacement list and uses the variable $repl$ to keep track of this list.
It then go over the entire input in reverse order and locate all the $VarTypeDecl$ objects
(the reverse traversal order almost always needs to used so 
that the replacement list is in the correct order). 
The code then extends the $repl$ list by pre-pending a new replacement (a (source, dest) tuple)
pair to it. After traversing the entire input, the list of replacements in $repl$ is then applied
collectively by invoking the $REPLACE$ operation on $input$.

\section {Other Tasks}
\paragraph {Value Numbering.} As value number is a combined process of hash table management
and AST modification, it is obvious that the POET associative map should be used to implement
the hashing of value numbers and variable names, and that AST modification should be implemented
by invoking the REPLACE operation. Note that {\bf you cannot modify ASTs in POET, see Section~\ref{sec-mod}). 
 
 \paragraph {Three-address or machine code generation}. This means translating the input program
 to another language. Here you need to define a collection of code templates for the new languages
 (in the case of three-address code, a single code template for formating three-address instructions). 
 The translation can then proceed by translating the input AST to a three-address AST (or machine
 code AST). The output AST can then be unparsed in the proper format using the POET output command, as illustrated in the following (see POET/examples/TranslatorDriver.pt).
 \begin {verbatim}
 <output to=(outputFile) syntax=(outputSyntax) from=resultCode/>
 \end {verbatim}
 

\end{document}
